%\chapter{Features of PltRedex}

\section{The Imp Language}

To introduce PLTRedex, the Imp language will be used. Its grammar can be seen in Figure \ref{imp-grammar}.

\begin{figure}[htb]
\begin{align*}
a &= X \text{ | } n \text{ | } a_1 + a_2 \text{ | } a_1 \times a_2 \\
b &= \textbf{true} \text{ | } \textbf{false} \text{ | } a1 \leq a2 \text{ | } \neg b \text{ | } b1 \land b2 \text{ | } b1 \lor b2 \\
c &= \textbf{skip} \text{ | } x = a \text{ | } c1; c2 \text{ | } \textbf{if } b \textbf{ then } c_1 \textbf{ else } c_2 \text{ | } \textbf{while } b \textbf{ do } c
\end{align*}
\caption{Grammar of Imp.}
\label{imp-grammar}
\end{figure}

\begin{itemize}
\item $a \in \text{ Aexp}$ is the set of all arithmetic expressions. $X$ ranges over location $Loc$; $n \in \mathbb{Z}$.
\item $b \in \text{ Bexp}$ is the set of all boolean expressions.
\item $c \in \text{ Com}$ is the set of all commands, including no-op, variable assignment, command sequencing, conditionals and while loops.
\item Finally set $\Sigma$ consists of functions $\sigma: Loc -> \mathbb{Z}$, mapping locations to integers. Thus, $\sigma(X)$ retrieves appropriate $n$ from location $X$.
\end{itemize}


\section{Structural Operational Semantics}
To be able to prove properties of programs, languages need to be described formally. Operational semantics describe the behavior of programming languages in the terms of the execution of programs on abstract machines. Structural operational semantics that were first introduced in ~\cite{plotkin}, represent computation using deductive systems meaning the execution of programs on abstract machines turns into a system of logical inferences. To define a language, inference rules are used such as ones seen in Figures \ref{infer-arith}, \ref{infer-bool}, \ref{infer-comm}.

\begin{figure}[htb]
\[
\begin{prooftree}
\infer0{
	\langle x, \sigma \rangle 
	\rightarrow  
	\langle \sigma(x), \sigma \rangle 
}
\end{prooftree} \; \; \; [\text{Variable}]
\]

\[
\begin{prooftree}
\hypo{
	\langle a_1, \sigma \rangle \rightarrow
	\langle a_1^\prime, \sigma \rangle 
} 
\infer1{
	\langle a_1 \oplus a_1, \sigma \rangle \rightarrow  
	\langle a_1^\prime \oplus a_1, \sigma \rangle
}
\end{prooftree} \; \oplus = \{+,\times\} \; \; \; [\text{AExp-Lhs-}\oplus]
\]

\[
\begin{prooftree}
\hypo{
	\langle a_1, \sigma \rangle \rightarrow
	\langle a_1^\prime, \sigma \rangle 
} 
\infer1{
	\langle n_1 \oplus a_1, \sigma \rangle \rightarrow  
	\langle n_1 \oplus a_1^\prime, \sigma \rangle
}
\end{prooftree} \; \oplus = \{+,\times\} \; \; \; [\text{AExp-Rhs-}\oplus]
\]

\[
\begin{prooftree}
\infer0{
	\langle n_1 \oplus n_1, \sigma \rangle \rightarrow  
	\langle n_3,  \sigma \rangle
}
\end{prooftree} \; \; \; \; \oplus = \{+,\times\}, n_1, n_1, n_3 \in \mathbb{Z} \; \; \; [\text{AExp-}\oplus]
\]


\caption{Evaluation of arithmetic expressions.}
\label{infer-arith}
\end{figure}
\begin{figure}[htb]
\[
\begin{prooftree}
\hypo{
	\langle a_1, \sigma \rangle \rightarrow
	\langle a_1^\prime, \sigma \rangle 
} 
\infer1{
	\langle a_1 \leq a_1, \sigma \rangle \rightarrow  
	\langle a_1^\prime \leq a_1, \sigma \rangle
}
\end{prooftree} \; \; \; [\text{BExp-Lhs-}\leq]
\]

\[
\begin{prooftree}
\hypo{
	\langle a_1, \sigma \rangle \rightarrow
	\langle a_1^\prime, \sigma \rangle 
} 
\infer1{
	\langle n_1 \leq a_1, \sigma \rangle \rightarrow  
	\langle n_1 \leq a_1^\prime, \sigma \rangle
}
\end{prooftree} \; \; \; [\text{BExp-Rhs-}\leq]
\]

\[
\begin{prooftree}
\infer0{
	\langle n_1 \leq n_1, \sigma \rangle \rightarrow  
	\langle t, \sigma \rangle
}
\end{prooftree} \; \; \; \; t \in \{ \textbf{true}, \textbf{false} \} \; \; \; [\text{BExp-}\leq]
\]

\[
\begin{prooftree}
\hypo{
	\langle b_1, \sigma \rangle \rightarrow
	\langle b_1^\prime, \sigma \rangle 
} 
\infer1{
	\langle b_1 \oslash b_1, \sigma \rangle \rightarrow  
	\langle b_1^\prime \oslash b_1, \sigma \rangle
}
\end{prooftree} \; \oslash = \{\land,\lor\} \; \; \; [\text{BExp-Lhs-}\oslash]
\]

\[
\begin{prooftree}
\hypo{
	\langle b_1, \sigma \rangle \rightarrow
	\langle b_1^\prime, \sigma \rangle 
} 
\infer1{
	\langle t_1 \oslash b_1, \sigma \rangle \rightarrow  
	\langle t_1 \oslash b_1^\prime, \sigma \rangle
}
\end{prooftree} \; \oslash = \{\land,\lor\} \; \; \; [\text{BExp-Rhs-}\oslash]
\]

\[
\begin{prooftree}
\infer0{
	\langle t_1 \oslash t_1, \sigma \rangle \rightarrow  
	\langle t_3,  \sigma \rangle
}
\end{prooftree} \; \; \; \; \oslash = \{\land,\lor\}, t_1, t_1, t_3 \in \{ \textbf{true}, \textbf{false} \}  \; \; \; [\text{BExp-}\oslash]
\]

\[
\begin{prooftree}
\hypo{
	\langle b_1, \sigma \rangle \rightarrow  
	\langle b_1^\prime,  \sigma \rangle
}
\infer1{
	\langle \neg b_1, \sigma \rangle \rightarrow  
	\langle \neg b_1^\prime,  \sigma \rangle
}
\end{prooftree} \; \; \; [\text{BExp-E}\neg]
\]

\[
\begin{prooftree}
\infer0{
	\langle \neg t_1, \sigma \rangle \rightarrow  
	\langle t_1,  \sigma \rangle
}
\end{prooftree} \; \; \; \; t_1, t_1 \in \{ \textbf{true}, \textbf{false} \}  \; \; \; [\text{BExp-}\neg]
\]



\caption{Evaluation of boolean expressions.}
\label{infer-bool}
\end{figure}

For arithmetic expressions in Imp, define evaluation relation $\langle a, \sigma \rangle \rightarrow \langle a^\prime, \sigma  \rangle$; meaning the evaluation of $a$ under state $\sigma$ in single step yields $a^\prime$. Figure \ref{infer-arith} shows seven distinct inference rules, with rules $[\text{Aexp-}\oplus]$, etc created for each operator in $\{+, \times \}$, and all operators having usual semantics w.r.t. $\mathbb{Z}$. These rules are specified in a way that ensures that all arithmetic expressions will be evaluated from left to right.


Similarly, for boolean expressions single-step evaluation relation $\langle b, \sigma \rangle \rightarrow \langle b^\prime, \sigma  \rangle$ is defined and related inference rules can be seen in Figure \ref{infer-bool}. Rules [$\text{BExp-}\oslash]$, etc, created for each operators $\land$ and $\lor$, as above.

Finally, a single-step evaluation relation for commands is defined: $\langle c, \sigma \rangle \rightarrow \langle c^\prime, \sigma^\prime \rangle$. Inference rules can been seen in Figure \ref{infer-comm}.

\begin{figure}[htb]
\[
\begin{prooftree}
\hypo{
	\langle a, \sigma \rangle \rightarrow
	\langle a^\prime, \sigma \rangle 
} 
\infer1{
	\langle x = a, \sigma \rangle \rightarrow  
	\langle x = a^\prime, \sigma \rangle
}
\end{prooftree} \; \; \; [\text{Assign-A}]
\]


\[
\begin{prooftree}
\infer0{
	\langle x = n, \sigma \rangle \rightarrow  
	\langle \textbf{skip},  \sigma[x \mapsto n] \rangle
}
\end{prooftree} \; \; \; [\text{Assign-N}]
\]

\[
\begin{prooftree}
\hypo{
	\langle c_0, \sigma \rangle \rightarrow
	\langle c_0^\prime, \sigma^\prime \rangle 
} 
\infer1{
	\langle c_0; c_1,  \sigma \rangle \rightarrow  
	\langle c_0^\prime; c_1, \sigma^\prime \rangle
}
\end{prooftree} \; \; \; [\text{Sequence-A}]
\]

\[
\begin{prooftree}
\infer0{
	\langle \textbf{skip}; c_1,  \sigma \rangle \rightarrow  
	\langle c_1, \sigma \rangle
}
\end{prooftree} \; \; \; [\text{Sequence-Skip}]
\]

\[
\begin{prooftree}
\hypo{
	\langle b, \sigma \rangle \rightarrow
	\langle b^\prime, \sigma \rangle 
} 
\infer1{
	\langle \textbf{if } b \textbf{ then } c_0 \textbf{ else } c_1, \sigma   \rangle
	\rightarrow
	\langle \textbf{if } b^\prime \textbf{ then } c_0 \textbf{ else } c_1, \sigma  \rangle
}
\end{prooftree} \; \; \; [\text{Conditional}]
\]

\[
\begin{prooftree}
\infer0{
\langle \textbf{if } \textbf{true} \textbf{ then } c_0 \textbf{ else } c_1, \sigma  \rangle
\rightarrow
\langle c_0, \sigma \rangle 
}
\end{prooftree} \; \; \; [\text{Conditional-True}]
\]

\[
\begin{prooftree}
\infer0{
\langle \textbf{if } \textbf{false} \textbf{ then } c_0 \textbf{ else } c_1, \sigma  \rangle
\rightarrow
\langle c_1, \sigma \rangle 
}
\end{prooftree} \; \; \; [\text{Conditional-False}]
\]

\[
\begin{prooftree}
\infer0{
\langle \textbf{while } b \textbf{ do } c, \sigma \rangle
\rightarrow
\langle \textbf{if } b \textbf{ then } (c; \textbf{while } b \textbf{ do } c)
\textbf{ else } \textbf{skip}, \sigma \rangle
}
\end{prooftree} \; \; \; [\text{While}]
\]

\caption{Inference rules for evaluation of commands.}
\label{infer-comm}
\end{figure}

To be able to sequence single-step evaluation steps, a reflexive transitive closure on relation $\langle c, \sigma \rangle \rightarrow \langle c^\prime, \sigma^\prime \rangle$ needs to be defined, as seen in Figure \ref{transitive-closure}.

\begin{figure}[htb]
\[
\begin{prooftree}
\infer0{
	\langle c, \sigma \rangle 
	\rightarrow^{*} \langle 
	c, \sigma \rangle
}
\end{prooftree} \; \; \; [\text{Reflexive}]
\]
\[
\begin{prooftree}
\hypo{
	\langle c, \sigma \rangle \rightarrow \langle c^\prime, \sigma^\prime \rangle
} 
\hypo{
	\langle c^\prime, \sigma^\prime \rangle \rightarrow^{*} \langle c^{\prime\prime}, \sigma^{\prime\prime} \rangle
}
\infer2{
	\langle c, \sigma \rangle \rightarrow^{*} \langle c^{\prime\prime}, \sigma^{\prime\prime} \rangle
}
\end{prooftree} \; \; \; [\text{Transitive}]
\]
\caption{Reflexive/transitive closure on single-step evaluation relation of commands.}
\label{transitive-closure}
\end{figure}


\section{Evaluation Contexts}
\label{01-evaluation-context}
As seen above, inference rules are quite verbose. PLTRedex uses \texttt{evaluation contexts}, originally introduced in ~\cite{felleisen1992revised}. Evaluation contexts are usually specified using a context free grammar, like in Figure \ref{infer-evaluation}.

An evaluation context $E[\bullet]$ is an expression containing a single occurrence of special symbol $\bullet$ called "hole". Given some term $t$ (abstract syntax tree of a program being evaluated), the goal is to decompose the term into some context $E$ and some expression $r$ that can be rewritten in some way; in other words, $t=E[r]$. Once such $r$ is found, expression $r$ is replaced with the hole: $E[\bullet]$. Then, the expression $r$ is reduced using some local reduction rule in one step obtaining expression $r^\prime$. Finally, $\bullet$ in $E[\bullet]$ gets replaced with $r^\prime$; $E[r^\prime]$ is the resulting expression. This way, only a single global general inference rule is required, as seen in Figure \ref{infer-evaluation}. 

\begin{figure}[htbp]
\begin{align*}
E = \bullet &\text{ | } E + a_1 \text{ | } n_1 + E \text{ | } E \times a_1 \text{ | } n_1 \times E \\
  &\text{ | } E \land b_1 \text{ | } t_1 \land E  \text{ | } E \lor b_1 \text{ | } t_1 \lor E \text{ | } \neg E \text{ | } E \leq a_1 \text{ | } n_1 \leq E  \\
  &\text{ | } x = E \text{ | } E;c \text{ | } \textbf{if } E \textbf{ then } c \textbf{ else } c
\end{align*}

\[
\begin{prooftree}
\infer0{
	\langle x, \sigma \rangle 
	\rightarrow  
	\langle \sigma(x), \sigma \rangle 
}
\end{prooftree} \; \; \; [\text{Variable}]
\]

\[
\begin{prooftree}
\infer0{
	\langle n_1 \oplus n_1, \sigma \rangle \rightarrow  
	\langle n_3,  \sigma \rangle
}
\end{prooftree} \; \; \; \; \oplus = \{+,\times\}, n_1, n_1, n_3 \in \mathbb{Z} \; \; \; [\text{AExp-}\oplus]
\]

\[
\begin{prooftree}
\infer0{
	\langle n_1 \leq n_1, \sigma \rangle \rightarrow  
	\langle t, \sigma \rangle
}
\end{prooftree} \; \; \; \; t \in \{ \textbf{true}, \textbf{false} \} \; \; \; [\text{BExp-}\leq]
\]

\[
\begin{prooftree}
\infer0{
	\langle t_1 \oslash t_1, \sigma \rangle \rightarrow  
	\langle t_3,  \sigma \rangle
}
\end{prooftree} \; \; \; \; \oslash = \{\land,\lor\}, t_1, t_1, t_3 \in \{ \textbf{true}, \textbf{false} \}  \; \; \; [\text{BExp-}\oslash]
\]

\[
\begin{prooftree}
\infer0{
	\langle \neg t_1, \sigma \rangle \rightarrow  
	\langle t_1,  \sigma \rangle
}
\end{prooftree} \; \; \; \; t_1, t_1 \in \{ \textbf{true}, \textbf{false} \}  \; \; \; [\text{BExp-}\neg]
\]

\[
\begin{prooftree}
\infer0{
	\langle x = n, \sigma \rangle \rightarrow  
	\langle \textbf{skip},  \sigma[x \mapsto n] \rangle
}
\end{prooftree} \; \; \; [\text{Assign-N}]
\]

\[
\begin{prooftree}
\infer0{
	\langle \textbf{skip}; c_1,  \sigma \rangle \rightarrow  
	\langle c_1, \sigma \rangle
}
\end{prooftree} \; \; \; [\text{Sequence-Skip}]
\]

\[
\begin{prooftree}
\infer0{
\langle \textbf{if } \textbf{true} \textbf{ then } c_0 \textbf{ else } c_1, \sigma  \rangle
\rightarrow
\langle c_0, \sigma \rangle 
}
\end{prooftree} \; \; \; [\text{Conditional-True}]
\]

\[
\begin{prooftree}
\infer0{
\langle \textbf{if } \textbf{false} \textbf{ then } c_0 \textbf{ else } c_1, \sigma  \rangle
\rightarrow
\langle c_1, \sigma \rangle 
}
\end{prooftree} \; \; \; [\text{Conditional-False}]
\]

\[
\begin{prooftree}
\infer0{
\langle \textbf{while } b \textbf{ do } c, \sigma \rangle
\rightarrow
\langle \textbf{if } b \textbf{ then } (c; \textbf{while } b \textbf{ do } c)
\textbf{ else } \textbf{skip}, \sigma \rangle
}
\end{prooftree} \; \; \; [\text{While}]
\]


\[
\begin{prooftree}
\hypo{
	\langle r, \sigma \rangle
	\rightarrow
	\langle r^\prime, \sigma^\prime \rangle
}
\infer1{
	\langle E[r], \sigma \rangle
	\rightarrow
	\langle E[r^\prime], \sigma^\prime \rangle
}
\end{prooftree} \; \; \; [\text{Generalized small-step}]
\]

\caption{Evaluation contexts, local reductions and generalized inference rule.}
\label{infer-evaluation}
\end{figure}

\section{PLTRedex: Term Rewriting}
PLTRedex provides a domain-specific language for specification of (1) a syntax of a language and (2) a series of term-rewriting rules based on the contextual operational semantics described in Section \ref{01-evaluation-context}. By applying the generalized small-step inference rule seen in Figure \ref{infer-evaluation} where applicable, and by transitivity (Figure \ref{transitive-closure}), a series of rewritten terms is obtained.

It is worth noting that these term-rewriting rules are not necessarily deterministic; that is, multiple reducible expressions may be found resulting in multiple rewritten terms.

\section{The Imp Language: PLTRedex Implementation}
\label{01-pltredex}
This section describes how Imp could be implemented using PLTRedex ~\cite{redexreference}. 

First, the grammar of the language needs to be defined. PLTRedex provides \texttt{define-language} form to do that, as seen in Figure \ref{imp-define-language}. Grammar is specified in EBNF-like manner. \texttt{define-language} consists of several \textit{non-terminal definitions}. Each non-terminal definition contains a number of \textit{patterns} that are matched against a term. 

\begin{figure}[H]
\begin{minted}[tabsize=2,obeytabs,escapeinside=||,mathescape=true,fontsize=\normalsize]{racket}
(define-language Imp
  (Aexp ::= var int (Aexp + Aexp) (Aexp * Aexp))
  (Bexp ::= bool (Aexp <= Aexp) (Bexp and Bexp) (Bexp or Bexp) (not Bexp))
  (Com  ::= skip
           (var = Aexp)
           (if Bexp Com Com ... else Com Com ...)
           (while Bexp Com Com ...))
  (var ::= variable-not-otherwise-mentioned)
  (int ::= integer)
  (bool ::= boolean) 

  (Loc ::= ((var int) ...))
  (Program ::= (Loc (Com ...)))

  (P ::= (var = E) (if E Com ... else Com ...) hole)
  (E ::= (E + Aexp) (int + E) (E * Aexp) (int * E) hole 
         (E <= Aexp) (int <= E)  (not E)     
         (E and Bexp) (bool and E)
         (E or  Bexp) (bool or  E)))
\end{minted}
\caption{Defining Imp in PLTRedex.}
\label{imp-define-language}
\end{figure}


\texttt{Aexp}, \texttt{Bexpr}, \texttt{Com} non-terminal definitions are equivalent to the grammar shown in Figure \ref{imp-grammar}. The only main difference is that \texttt{Com} does not model command sequencing as described in the grammar but utilizes so-called ellipsis (\texttt{...}) - patterns under ellipsis such as \texttt{Com ...} which match zero or more terms that match the \texttt{Com} pattern.

\texttt{var}, \texttt{int}, \texttt{bool} patterns match built-in patterns provided by PLTRedex - \texttt{integer} that matches integers, \texttt{boolean} that matches booleans and \texttt{variable-not-otherwise-mentioned} that matches any variable not used in the language definition such as \texttt{+} or \texttt{<=}.

\texttt{Loc} non-terminal represents locations and is modelled as a list containing zero or more variable-integer pairs.

\texttt{Program} non-terminal models $\langle c, \sigma \rangle$; that is, locations and potentially empty list of commands.

Non-terminals \texttt{P} and \texttt{E} specify evaluation contexts $E$, as described in Section \ref{01-evaluation-context}. \texttt{hole} represents hole $\bullet$.


Now, the operations $\sigma(X)$ and $\sigma[X \mapsto n]$ that retrieve from and store integers in provided locations need to be implemented. PLTRedex provides \textit{metafunctions}; that is, functions that act on terms and produce terms by rewriting them in some way. Definitions of two metafunctions can be seen in Figure \ref{imp-define-mf}. Metafunctions essentially consist of pattern-term clauses. A provided term is matched against the pattern. Upon successful matching appropriate terms get assigned to pattern-variables such as \texttt{var\_1} or \texttt{int\_2}, and they get plugged into the term. If matching is unsuccessful, the next clause is matched. An error is raised if none of the clauses match the provided term. All non-terminals in patterns are resolved with respect to the language - \texttt{Imp}.

Finally, all terms provided as arguments and all terms that are produced by metafunctions must satisfy the contract. In case of the first metafunction, the contract is \texttt{var-lookup : Loc var -> int}. \texttt{var-lookup} is the name of the function. Following the colon, provided argument terms must match the patterns \texttt{Loc} and \texttt{var}, otherwise it is an error. Similarly, all resulting terms produced by the \texttt{var-lookup} metafunction must match the pattern \texttt{int} which is preceded by the symbol \texttt{->}.

Careful readers might notice that the pattern of the first clause in \texttt{var-lookup} has two occurrences of the pattern-variable \texttt{var}. This means that both terms bound to both occurrences after pattern-matching must be exactly the same.

\begin{figure}[htb]
\begin{minted}[tabsize=2,obeytabs,escapeinside=||,mathescape=true,fontsize=\normalsize]{racket}
(define-metafunction Imp
  var-lookup : Loc var -> int
  [(var-lookup ((var_1 int_1) ... (var int) (var_2 int_2) ...) var) int]
  [(var-lookup ((var_1 int_1) ...) var) 0])

(define-metafunction Imp
    var-assign : Loc var int -> Loc 
  [(var-assign ((var_1 int_1) ... (var int_3) (var_2 int_2) ...) var int)
   ((var_1 int_1) ... (var int) (var_2 int_2) ...)]
  [(var-assign ((var_1 int_1) ...) var int)
   ((var_1 int_1) ... (var int))])
\end{minted}
\caption{Metafunctions for retrieving and storing integers.}
\label{imp-define-mf}
\end{figure}

Finally, evaluation relation needs to be specified. PLTRedex utilizes evaluation contexts as described in Section \ref{01-evaluation-context}. Reduction relation can be seen in Figure \ref{imp-define-red} and is defined using the \texttt{reduction-relation} form. Similarly to metafunctions, it contains a series of reduction cases, each consisting of a pattern and a term. It may be the case that multiple reduction cases may apply to the provided term, indicating non-deterministic behavior of a language.

The \texttt{(in-hole pattern1 pattern2)} pattern and \texttt{(in-hole term1 term2)} term implements evaluation contexts as described in Section \ref{01-evaluation-context}. The term is recursively traversed to find the redex matching \texttt{pattern2}. Once it has been found, it is replaced with \texttt{hole} and the term is matched against \texttt{pattern1} to ensure that it is indeed a valid evaluation context. After matching of redex and context, redex is rewritten according to local inference rules. They can be implemented through metafunctions or by calling Racket functions, like it is done for addition and multiplication operations. Finally, after reducing the term \texttt{term2}, it is plugged back into the context \texttt{term1}, thus replacing the \texttt{hole}.

All terms that are arguments to the \texttt{reduction-relation} and all resulting terms after applying \texttt{reduction-relation} must match the pattern \texttt{Program}.


\begin{figure}[H]
\begin{minted}[tabsize=2,obeytabs,escapeinside=||,mathescape=true,fontsize=\normalsize]{racket}
(apply-reduction-relation* imp-red 
(term (() {
	(i = 0)
	[while (i <= 1)
		[if (i <= 0) (y = 1335) else (z = 1)]
		(i = (i + 1))]
})))

; '((((i 2) (y 1335) (z 1)) ()))
\end{minted}
\caption{Applying reduction relation to a term.}
\label{imp-red-apply}
\end{figure}

\begin{figure}[htb]
\begin{minted}[tabsize=2,obeytabs,escapeinside=||,mathescape=true,fontsize=\normalsize]{racket}
(define imp-red 
  (reduction-relation Imp
  #:domain Program 
    [--> (Loc ((while Bexp Com_1 ...) Com_2 ...))
         (Loc ((if Bexp Com_1 ... (while Bexp Com_1 ...) 
                else skip) Com_2 ...))
         "while"]
    [ --> (Loc ((in-hole P (var = int)) Com ...))
          ((var-assign Loc var int) (Com ...))
          "var-assign"]
    [ --> (Loc (skip Com ...))
          (Loc (Com ...))
          "skip"]
    [--> (Loc ((in-hole P var) Com ...))
         (Loc ((in-hole P (var-lookup Loc var)) Com ...))
         "var-lookup"]
    [--> (Loc ((in-hole P (if #t Com_1 ... else Com_2 ...)) Com ...))
         (Loc (Com_1 ... Com ...))
         "if-true"]
    [--> (Loc ((in-hole P (if #f Com_1 ... else Com_2 ...)) Com ...))
         (Loc (Com_2 ... Com ...))
         "if-false"]
    [--> (Loc ((in-hole P (int_1 + int_2)) Com ...))
         (Loc ((in-hole P ,(+ (term int_1) (term int_2))) Com ...))
         "integer-add"]
    [--> (Loc ((in-hole P (int_1 * int_2)) Com ...))
         (Loc ((in-hole P ,(* (term int_1) (term int_2))) Com ...))
         "integer-mul"]
    [--> (Loc ((in-hole P (bool_1 and bool_2)) Com ...))
         (Loc ((in-hole P ,(and (term bool_1) (term bool_2))) Com ...))
         "boolean-and"]
    [--> (Loc ((in-hole P (bool_1 or bool_2)) Com ...))
         (Loc ((in-hole P ,(or (term bool_1) (term bool_2))) Com ...))
         "boolean-or"]
    [--> (Loc ((in-hole P (not bool_1)) Com ...))
         (Loc ((in-hole P ,(not (term bool_1))) Com ...))
         "boolean-not"]
    [--> (Loc ((in-hole P (int_1 <= int_2)) Com ...))
         (Loc ((in-hole P ,(<= (term int_1) (term int_2))) Com ...))
         "less-equal"]))
\end{minted}
\caption{Reduction relation for Imp.}
\label{imp-define-red}
\end{figure}

Finally, the reduction-relation defined above is applied to some term using a \newline \texttt{apply-reduction-relation} function, as seen in Figure \ref{imp-red-apply}

