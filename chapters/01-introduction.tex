\chapter{Introduction}
\section{Motivation}
PltRedex is domain-specific language, embedded into Racket, designed for specifying and debugging operational semantics. By specifying a language and reduction rules, PltRedex is able to apply reduction rules to terms by rewriting them. 
However, after testing the PltRedex specification one might want to turn it into a stand-alone interpreter and that's when dependency on Racket turns into a problem for the following reasons:

\begin{enumerate}
\item 
One has to ship the entire Racket runtime to be able to run the specification.
\item
Racket is not particularly fast. 
\end{enumerate}

PyPltRedex is the tool that attempts to solve the problem by taking the PltRedex specification and compiling it into the RPython language - a statically analyzable subset of Python programming language that is used for the implementation of interpreters within the PyPy toolchain. Using RPython toolchain, the resulting RPython program is then compiled into a stand-alone, more efficient and considerably more low-level version of the program. However, by removing Racket from the equation, the specification has to be modified to use RPython instead.

\section{Goals}
Since PltRedex has existed for a while and provides a wide range of functionality ranging from language specification, type checking and testing, PyPltRedex implements a tiny subset of PltRedex that is enough to be usable. In particular,

\begin{itemize}
\item Only a subset of pattern language provided by PltRedex is supported.
\item
Only functionality required for term rewriting is supported.
\end{itemize}

\section{Thesis Outline}
This thesis consists of ten chapters, excluding introduction.

Chapter 2 provides an overview of what PLTRedex is, describes programming language called \texttt{Imp}, provides small-step operational semantics of \texttt{Imp} and explains how \texttt{Imp} is implemented using PLTRedex.

Chapter 3 provides description of what PyPy framework and RPython toolchain is, as well as it describes compile-time representation of RPython employed by PyPltRedex.

Chapter 4 describes runtime library of PyPltRedex. Runtime representation of terms and matches is described, how terms are parsed is explained.

Chapter 5 describes compile-time representation of the pattern language provided by PLTRedex, of terms and of top-level forms.

Chapter 6 describes how patterns, terms, and other forms are preprocessed and analyzed.

Chapter 7 describes how RPython code is generated for terms, patterns and other top-level forms.

Chapter 8 describes testing methodology and provides evaluation. 

Chapter 9 covers the work that is still needed to be done.

