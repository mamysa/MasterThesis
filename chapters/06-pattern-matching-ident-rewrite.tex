\section{Transformation: Identifier Rewriting}

\subsection{Motivation}

Recall that for any pattern in $DefineLanguage$ constraint-checking is not performed. This means that the suffix is not relevant - one can write pattern \texttt{(+ e e)} as \texttt{(+ e\_1 e\_1)} or \texttt{(e\_1 e\_2)}. However, since `Match` objects cannot store more than one assignment for pattern variable, patterns like \texttt{(+ e e)} have to rewritten. PyPltRedex solves this problem by first stripping the suffix and then generating a fresh one for each pattern variable.

\subsection{Algorithm}
Givem $DefineLanguage$, each pattern in $NtDefinition$ is visited recursively.

\begin{itemize}
\item Given \BuiltInPattern, let $p=prefix(s)$, and then return new \BuiltInPattern with $s= fresh(p)$.
\item Given \Nt, return new $\Nt$ with $s=fresh(nt)$
\end{itemize}

\subsection{Example}

Figure \ref{id-rewrite-example} shows an example of the transformation on \DefineLanguage form. To all non-terminals \texttt{e}, \texttt{n} and built-in pattern \texttt{number} numerical suffixes are added.

\begin{figure}[h]
	\begin{minipage}{0.5\linewidth}
		\centering
		\begin{minted}[tabsize=2,obeytabs,escapeinside=||,mathescape=true,fontsize=\small]{Racket}
(define-language L
	(e ::= (+ |\colorbox{identbefore}{e} \colorbox{identbefore}{e}|) 
	       (* |\colorbox{identbefore}{e} \colorbox{identbefore}{e}|) |\colorbox{identbefore}{n}|)
	(n ::= |\colorbox{identbefore}{number}|))
		\end{minted}
	\end{minipage}
	\begin{minipage}{0.5\linewidth}
		\centering
		\begin{minted}[tabsize=2,obeytabs,escapeinside=||,mathescape=true,fontsize=\small]{Racket}
(define-language L
	(e ::= (+ |\colorbox{identafter}{e\_0} \colorbox{identafter}{e\_1}|) 
	       (* |\colorbox{identafter}{e\_2} \colorbox{identafter}{e\_3}|) |\colorbox{identafter}{n\_0}|)
	(n ::= |\colorbox{identafter}{number\_0}|))
		\end{minted}
	\end{minipage}
	\caption{\texttt{define-language} before and afte renaming of identifiers.}
	\label{id-rewrite-example}
\end{figure}


