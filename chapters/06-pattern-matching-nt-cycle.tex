\section{DefineLanguage: Non-Terminal Cycle Checking}

\subsection{Motivation}
PltRedex doesn't allow language definitions such as the one in Figure \ref{dl-ntcyclegraph}.

\begin{figure}[H]
\begin{minipage}{0.45\linewidth}
	\centering
\begin{minted}[tabsize=1,obeytabs,escapeinside=||,mathescape=true,linenos,fontsize=\small]{racket}
(define-language L
	(e ::= (e e) n)
	(n ::= (n n) number p)
	(p ::= (p p) real e))
	(s ::= string)
\end{minted}
\end{minipage}
\begin{minipage}{0.45\linewidth}
	\centering
	\includegraphics[scale=0.18]{transformation-pattern-ntgraph.png}
\end{minipage}
	\caption{\texttt{define-language} and its non-terminal graph}
	\label{dl-ntcyclegraph}
\end{figure}

The problem with the language above (aside from it being completely useless) is a cycle of non-terminals $n \rightarrow p \rightarrow e$. When testing if some term is a non-terminal, a term is matched against every pattern in a non-terminal definition. For example, given term \texttt{String("hello world")}, it is matched against non-terminal \texttt{e}. Since  the term doesn't match the pattern \texttt{(e e)}, non-terminal \texttt{n} is then matched. Since the term doesn't match any patterns here either, non-terminal \texttt{p} is then matched. Similarly, the term doesn't match any patterns here either, \texttt{e} is then matched, but that's where the matching has started and thus \textit{infinite recursion} becomes an issue. Languages need to be analyzed for presence of non-terminal cycles.

\subsection{Algorithm}
To detect such cycles, \DefineLanguageNoArg \space needs to be interpreted as a directed graph and some cycle-detecting algorithm must be used. The graph is constructed in the following manner.

\begin{enumerate}
\item
For each \NtDefinitionN[n], create a vertex labeled $n$.
\item
For $p_1, ..., p_n$, if $p_i$ is \NonTerminal[m][m\_?], create the edge $n\rightarrow m$.
\end{enumerate}


To detect cycles, depth-first-search is employed. Vertices in the graph can be assigned one of three colors:

\begin{itemize}
\item

\textbf{White} - meaning the vertex hasn't been visited before. All vertices are initially assigned this color.

\item
\textbf{Gray} - meaning successors of the vertex $v$ are still being visited. When $v$ is visited for the first time, $color(v)$ becomes \textbf{Gray}.
\item
\textbf{Black} - meaning all successors of the vertex $v$ have been explored.
\end{itemize}

A path of vertices visited during depth-first traversal is maintained.  Let $V$ be a set of vertices whose \DFSColor{$v$}{Black}, initially empty; let $U$ be a set of vertices whose \DFSColor{$v$}{White}, initially containing all vertices of the graph.

The algorithm proceeds in the following way:

\begin{enumerate}
\item Pick random vertex $u \in U$.
\item If \DFSColor{$u$}{White}, set color to \textbf{Gray} and add $u$ to the path.
\item If \DFSColor{$u$}{Gray}, report the cycle (using vertices on the path) and abort compilation.
\item Visit each successor vertex $u^{\prime}$ of $u$ recursively.
\item Set color of $u$ to \textbf{Black} and remove it from the path. $V = V \cup \{u\}$.
\item If there are no more vertices left to visit, let $U = U-V$ and go to step 1.
\end{enumerate}

It should be noted that the algorithm described does not report all the cycles in the graph but the first one it manages to find. One improvement could be finding all cycles in the graph.
