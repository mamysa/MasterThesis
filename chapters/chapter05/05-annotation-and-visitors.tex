\section{Visitors}
\label{section:visitors}
Since patterns, term-templates and top-level forms require analysis or transformations applied to them, visitors for each \texttt{TermTemplate}, \texttt{Pattern} and \texttt{TopLevelForm} are provided, as seen in Figure \ref{class-diagram-visitors}.

In other programming languages the \texttt{Visitor} design pattern would require each \texttt{TermTemplate}, \texttt{Pattern} or \texttt{TopLevelForm} class to implement the \texttt{accept} method with \texttt{Visitor} passed as parameter. However, since Python is a dynamic language, it is possible to invoke methods given their names as strings. The \texttt{\_visit} method does exactly that - it looks up the type-name of a passed element, constructs the string, attempts to retrieve the method, and calls it. For example, given the \texttt{PatternSequence} instance, the resulting method name would be \texttt{\_visitPatternSequence}.

The only method of interest left is \texttt{run}. It is expected to be overridden by each transformation or analysis pass.  Since patterns used in \texttt{define-language} form require different treatment, \texttt{run} implementation may contain iteration logic over \texttt{define-language}, calling \texttt{\_visit} on each pattern.

\section{\texttt{Annotatable} Class}

When applying transformations/analyses passes to top-level forms, patterns, and term-template, additional information needs to be stored pertaining to those elements. One more obvious idea is to store those bits of information in a separate hash-map or dictionary, with \texttt{TermTemplate}, \texttt{Pattern} or \texttt{TopLevelForm} elements acting as keys. Unfortunately, the problem with this approach is that transformation passes may completely replace these elements, making keys in the dictionary invalid.

PyPltRedex solves this by storing these bits of information directly into the nodes. \texttt{TermTemplate}, \texttt{Pattern}, and \texttt{TopLevelForm} are subclasses of the \texttt{Annotatable} class. The class diagram can be seen in Figure \ref{class-diagram-visitors}.

Python's dynamicity is leveraged to store arbitrary information in the dictionary with \texttt{str} instances acting as keys. \texttt{addmetadata} adds data to the dictionary, \texttt{getmetadata} retrieves it from the dictionary. \texttt{removemetadata} removes key-value pair from the dictionary. \texttt{copymetadatafrom} makes a shallow copy of the dictionary of the provided element, if the element is being replaced by something else.

\begin{figure}[ht]
	\centering
	\makebox[\textwidth][c] { \includegraphics[scale=0.18]{class-diagram-visitors.png} }
	\caption{Representation of visitors and \texttt{Annotatable} class.}
\label{class-diagram-visitors}
\end{figure}

\FloatBarrier
