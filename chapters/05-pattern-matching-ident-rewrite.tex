\section{Transformation: Identifier Rewriting}

\subsection{Motivation}

Recall that for any pattern in $DefineLanguage$ constraint-checking is not performed. This means that the suffix is not relevant - one can write pattern \texttt{(+ e e)} as \texttt{(+ e\_1 e\_1)} or \texttt{(e\_1 e\_2)}. However, since `Match` objects cannot store more than one assignment for pattern variable, patterns like \texttt{(+ e e)} have to rewritten. PyPltRedex solves this problem by first stripping the suffix and then generating a fresh one for each pattern variable.

\subsection{Algorithm}
Pattern is visited recursively.

\begin{itemize}
\item Given \BuiltInPattern, let $p=prefix(s)$, and then $fresh(p)$.
\item Given \Nt, $fresh(nt)$
\end{itemize}

\subsection{Example}
TODO
