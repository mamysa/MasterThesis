\chapter{Features of PltRedex}

\section{Imp}

To introduce PLTRedex, Imp language will be used, originally introduced by Glynn Winskel.

\begin{lstlisting}
a = X | n | a1 + a2 | a1 * a2
b = true | false | a1 <= a2 | not b | b1 and b2 | b1 or b2
c = skip | x = a | c1; c2 | if b c1 else c2 | while b c


a in Aexp arithmetic expressions
b in Bexp boolean expressions
c in Com  commands 
X in Loc  locations
n in Z
\end{lstlisting}

\begin{figure}[h]
\[
\begin{prooftree}
\infer0{
	\langle x, \sigma \rangle 
	\rightarrow  
	\langle \sigma(x), \sigma \rangle 
}
\end{prooftree} \; \; \; [\text{Variable}]
\]

\[
\begin{prooftree}
\hypo{
	\langle a_1, \sigma \rangle \rightarrow
	\langle a_1^\prime, \sigma \rangle 
} 
\infer1{
	\langle a_1 \oplus a_2, \sigma \rangle \rightarrow  
	\langle a_1^\prime \oplus a_2, \sigma \rangle
}
\end{prooftree} \; \oplus = \{+,\times\} \; \; \; [\text{AExp-Lhs-}\oplus]
\]

\[
\begin{prooftree}
\hypo{
	\langle a_2, \sigma \rangle \rightarrow
	\langle a_2^\prime, \sigma \rangle 
} 
\infer1{
	\langle a_1 \oplus a_2, \sigma \rangle \rightarrow  
	\langle a_1 \oplus a_2^\prime, \sigma \rangle
}
\end{prooftree} \; \oplus = \{+,\times\} \; \; \; [\text{AExp-Rhs-}\oplus]
\]

\[
\begin{prooftree}
\infer0{
	\langle n_1 \oplus n_2, \sigma \rangle \rightarrow  
	\langle n_3,  \sigma \rangle
}
\end{prooftree} \; \; \; \; \oplus = \{+,\times\}, n_3 \in \mathbb{Z} \; \; \; [\text{AExp-}\oplus]
\]


\caption{Evaluation of arithmetic expressions.}
\end{figure}

\begin{figure}[h]
\[
\begin{prooftree}
\hypo{
	\langle a_1, \sigma \rangle \rightarrow
	\langle a_1^\prime, \sigma \rangle 
} 
\infer1{
	\langle a_1 \leq a_2, \sigma \rangle \rightarrow  
	\langle a_1^\prime \leq a_2, \sigma \rangle
}
\end{prooftree} \; \; \; [\text{BExp-Lhs-}\leq]
\]

\[
\begin{prooftree}
\hypo{
	\langle a_2, \sigma \rangle \rightarrow
	\langle a_2^\prime, \sigma \rangle 
} 
\infer1{
	\langle n_1 \leq a_2, \sigma \rangle \rightarrow  
	\langle n_1 \leq a_2^\prime, \sigma \rangle
}
\end{prooftree} \; \; \; [\text{BExp-Rhs-}\leq]
\]

\[
\begin{prooftree}
\infer0{
	\langle n_1 \oslash n_2, \sigma \rangle \rightarrow  
	\langle t, \sigma \rangle
}
\end{prooftree} \; \; \; \; t \in \{ \textbf{true}, \textbf{false} \} \; \; \; [\text{BExp-}\leq]
\]

\[
\begin{prooftree}
\hypo{
	\langle b_1, \sigma \rangle \rightarrow
	\langle b_1^\prime, \sigma \rangle 
} 
\infer1{
	\langle b_1 \oslash b_2, \sigma \rangle \rightarrow  
	\langle b_1^\prime \oslash b_2, \sigma \rangle
}
\end{prooftree} \; \oslash = \{\land,\lor\} \; \; \; [\text{BExp-Lhs-}\oslash]
\]

\[
\begin{prooftree}
\hypo{
	\langle b_2, \sigma \rangle \rightarrow
	\langle b_2^\prime, \sigma \rangle 
} 
\infer1{
	\langle t_1 \oslash b_2, \sigma \rangle \rightarrow  
	\langle t_1 \oslash b_2^\prime, \sigma \rangle
}
\end{prooftree} \; \oslash = \{\land,\lor\} \; \; \; [\text{BExp-Rhs-}\oslash]
\]

\[
\begin{prooftree}
\infer0{
	\langle t_1 \oslash t_2, \sigma \rangle \rightarrow  
	\langle t_3,  \sigma \rangle
}
\end{prooftree} \; \; \; \; \oslash = \{\land,\lor\}, t_1, t_2, t_3 \in \{ \textbf{true}, \textbf{false} \}  \; \; \; [\text{BExp-}\oslash]
\]

\[
\begin{prooftree}
\hypo{
	\langle b_1, \sigma \rangle \rightarrow  
	\langle b_1^\prime,  \sigma \rangle
}
\infer1{
	\langle \neg b_1, \sigma \rangle \rightarrow  
	\langle \neg b_1^\prime,  \sigma \rangle
}
\end{prooftree} \; \; \; [\text{BExp-E}\neg]
\]

\[
\begin{prooftree}
\infer0{
	\langle \neg t_1, \sigma \rangle \rightarrow  
	\langle t_2,  \sigma \rangle
}
\end{prooftree} \; \; \; \; \oslash = \{\land,\lor\}, t_1, t_2 \in \{ \textbf{true}, \textbf{false} \}  \; \; \; [\text{BExp-}\neg]
\]



\caption{Evaluation of boolean expressions.}
\end{figure}

\begin{figure}[h]
\[
\begin{prooftree}
\hypo{
	\langle a, \sigma \rangle \rightarrow
	\langle a^\prime, \sigma \rangle 
} 
\infer1{
	\langle x = a, \sigma \rangle \rightarrow  
	\langle x = a^\prime, \sigma \rangle
}
\end{prooftree} \; \; \; [\text{Assign-A}]
\]


\[
\begin{prooftree}
\infer0{
	\langle x = n, \sigma \rangle \rightarrow  
	\langle \textbf{skip},  \sigma[x \mapsto n] \rangle
}
\end{prooftree} \; \; \; [\text{Assign-N}]
\]

\[
\begin{prooftree}
\hypo{
	\langle c_0, \sigma \rangle \rightarrow
	\langle c_0^\prime, \sigma^\prime \rangle 
} 
\infer1{
	\langle c_0; c_1,  \sigma \rangle \rightarrow  
	\langle c_0^\prime; c_1, \sigma^\prime \rangle
}
\end{prooftree} \; \; \; [\text{Sequence-A}]
\]

\[
\begin{prooftree}
\infer0{
	\langle \textbf{skip}; c_1,  \sigma \rangle \rightarrow  
	\langle c_1, \sigma \rangle
}
\end{prooftree} \; \; \; [\text{Sequence-Skip}]
\]

\[
\begin{prooftree}
\hypo{
	\langle b, \sigma \rangle \rightarrow
	\langle b^\prime, \sigma \rangle 
} 
\infer1{
	\langle \textbf{if } b \textbf{ then } c_0 \textbf{ else } c_1, \sigma   \rangle
	\rightarrow
	\langle \textbf{if } b^\prime \textbf{ then } c_0 \textbf{ else } c_1, \sigma  \rangle
}
\end{prooftree} \; \; \; [\text{Conditional}]
\]

\[
\begin{prooftree}
\infer0{
\langle \textbf{if } \textbf{true} \textbf{ then } c_0 \textbf{ else } c_1, \sigma  \rangle
\rightarrow
\langle c_0, \sigma \rangle 
}
\end{prooftree} \; \; \; [\text{Conditional-True}]
\]

\[
\begin{prooftree}
\infer0{
\langle \textbf{if } \textbf{false} \textbf{ then } c_0 \textbf{ else } c_1, \sigma  \rangle
\rightarrow
\langle c_1, \sigma \rangle 
}
\end{prooftree} \; \; \; [\text{Conditional-False}]
\]

\[
\begin{prooftree}
\infer0{
\langle \textbf{while } b \textbf{ do } c, \sigma \rangle
\rightarrow
\langle \textbf{if } b \textbf{ then } (c; \textbf{while } b \textbf{ do } c)
\textbf{ else } \textbf{skip}, \sigma \rangle
}
\end{prooftree} \; \; \; [\text{While}]
\]

\caption{Evaluation of commands.}
\end{figure}



~\cite{plotkin}


Set of states $\Sigma$ consists of of functions $\sigma$: \texttt{Loc -> Z}. Thus, $\sigma(X)$ is contents of location $X$ in state $\sigma$. 

Given arithmetic expression $a$ and state $\sigma$, define evaluation relation $<a,\sigma> \rightarrow n$: arithmetic expression $a$ in state $\sigma$ evaluates to integer $n$.

Similarly, define evaluation relation for boolean expressions $b$. $<b, \sigma> \rightarrow {true, flase}$: boolean expression $b$ in state $\sigma$ evaluates to either \texttt{true} or \texttt{false}.

Finally, define evaluation relation for command $c$. Given $c$ and state $\sigma$, $<c, \sigma> -> \sigma\prime$. 

Small-step operational semantics are specified below using inference rules.

TODO-this is quite lengthy. Also provide derivation tree for some example - say \texttt{x = 12; if x <= 20 x = x + 1 else x = x - 1 } showing resulting state after evaluation.


\section{Evaluation Contexts}

Inference rules above are verbose and can be tricky to evaluate. PLTRedex uses \texttt{evaluation contexts}, originally introduced in \texttt{https://www2.ccs.neu.edu/racket/pubs/tcs92-fh.pdf}. 

An evaluation context $E[.]$ is an expression containing a single occurence of special symbol $.$ called "hole".$E[e]$ is the expression that results from replacing $.$ with some expression $e$. Evaluation contexts are usually specified using EBNF grammar. 

Using evaluation contexts, the following inference is introduced

$(r, \sigma) \rightarrow (t\prime, \sigma) THEN (E[r], \sigma\prime) \rightarrow (E[t\prime], \sigma\prime)$. Where redex $r$ reduces to some expression $r$ in a single step.

This allows for specification one one global reduction rule seen above and reduction rule for each redex $r$. These are shown in Figure TODO.

\begin{lstlisting}

(x, sigma) -> (n, sigma) where n = sigma(x)
(n1 + n2, sigma) -> (n, sigma) where n = n1 + n2, n1, n2, n in Z.
(b1 and b2, sigma) -> (b, sigma) where b = b1 and b2.
(skip;c, sigma) -> (c, sigma)

E = []
  | (x = E)
  | E;c
  | (E + e)
  | (n + E)
  | (E * e)
  | (n + E)
  | if E c else c
 
TODO complete this.
\end{lstlisting}


\subsection{Imp: PLTRedex Implementation}

Basically include \texttt{https://github.com/mamysa/PyPltRedex/blob/master/examples/imp/imp-racketcompat.rkt} here but explain all the forms.

Also mention non-determinism.

