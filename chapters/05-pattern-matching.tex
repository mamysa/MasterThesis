\chapter{Pattern Matching}

\section{Pattern Language}

This section describes subset of PltRedex's pattern specification language supported by \texttt{PyPltRedex}. Grammar for the pattern language can be seen below, in EBNF notation. 

\begin{lstlisting}
pattern = number 
        | integer 
		| real 
		| natural 
		| string 
		| boolean 
		| variable-not-otherwise-mentioned 
		| hole 
		| symbol
        | (in-hole pattern pattern)
        | (pattern-sequence *) 
pattern-sequence : pattern 
                 | pattern ...  # literal ellipsis
\end{lstlisting}

\begin{itemize}
\item
\textit{number} pattern matches any number.

\item
\textit{integer} matches any exact integer. 

\item
\textit{real} matches any real number.

\item
\textit{natural} matches any natural number; that is any non-negative integer.

\item
\textit{string} matches any string.

\item
\textit{boolean} matches any boolean - \texttt{\#t} or \texttt{\#f}.

\item
\textit{variable-not-otherwise-mentioned} matches any symbol that is not used as a literal in language definition. For example, if language definition contains pattern \texttt{(+ number number)} \textit{variable-not-otherwise-mentioned} will not match symbol \texttt{+}.

\item
\textit{hole} matches \texttt{hole} term exactly.

\item
\textit{symbol} matches any symbol except if its value coincides with non-terminal symbol in language definition.
\end{itemize}

All pattern above except \textit{hole} can be suffixed with underscore and identifier (for example, \textit{number\_1}) to create binding to matched term.

\begin{itemize}
\item
\textit{(in-hole pattern pattern)} traverses the term trying to match the second pattern; upon successful match the term matching the second pattern is replaced with term `hole` and then the first pattern is matched. First pattern must match exactly one hole.

\item

\textit{pattern-sequence} pattern matches a term list, where each pattern-sequence element matches an element of the list. Each individual pattern within the sequence can be suffixed with \texttt{...} (literal ellipsis) and that will match zero or more terms matching the pattern.
\end{itemize}

If patterns in the pattern-sequence are suffixed with the same identifier (e.g. \texttt{(number\_1 number\_1)})), then the match is contrained to terms that are equal. That means term \texttt{(1 1)} matches the pattern but \texttt{(1 2)} does not. For patterns in \textit{define-language} constraint checking is not performed. PltRedex provides other constraint checks but they will not be considered.
