\section{Term-Templates}

\subsection{Term Literals}

Given \TermLiteral, return the new term instance based on \texttt{TermLiteralKind}, as defined in Section \ref{section:term-templates}. Term-templates of this kind are guaranteed to not have \texttt{ForEach} and \texttt{InArg} annotations, by construction.

\begin{figure}[ht]
\begin{minted}[tabsize=2,obeytabs,escapeinside=||,mathescape=true,fontsize=\normalsize]{python}

	
def |$f_t$|(match):
	
	return Variable(|$v$|)
	
	return Integer(|$v$|)
	
	return Float(|$v$|)
	
	return String(|$v$|)
	
	return Boolean(|$v$|)
	
	return Hole()
	

\end{minted}
\caption{Code generation form literal terms.}
\label{codegen-term-template-lit}
\end{figure}


\subsection{Pattern Variables}

Given $t =$ \PatternVariable, let $f_t$ be a term-generating function for $t$. One of two scenarios may happen:
\begin{itemize}
\item $t$ has an \texttt{InArg(paramname)} annotation, meaning that this pattern-variable originally must have been under ellipsis.
\item $t$ has an \texttt{ReadMatch} annotation, meaning that this pattern-variable must not have been under an ellipsis. Simply return the term bound to pattern-variable $s$.
\end{itemize}

\begin{figure}[htb]
\begin{minted}[tabsize=2,obeytabs,escapeinside=||,mathescape=true,fontsize=\normalsize]{python}

	
	
	
	
def |$f_t$|(match, |$arg_1$|):
	return |$arg_1$|
	
def |$f_t$|(match):
	return match.getbinding(|$s(rm_1)$|)
	

\end{minted}
\caption{Code generation for pattern-variables.}
\label{codegen-term-pv}
\end{figure}

\subsection{\TermSequenceNoArg}

Given \TermSequence, recursively generate code for term-templates $t_i$, resulting in functions $f_{t_1},..., f_{t_n}$. Let $f_t$ be a term-generating function for $t$. Let $arg_1, ..., arg_n$ be a list of parameters to $f_t$, acquired by reading \texttt{InArg} annotation. Let $mr_1, ..., mr_n$ be a list of \texttt{(id: str, s: str)} acquired by reading \texttt{ReadMatch} annotation. For each such tuple, emit assignment to the metavariable \texttt{id} to the result of calling \texttt{match.getbinding(s)}.


\begin{figure}[ht]
\begin{minted}[tabsize=2,obeytabs,escapeinside=||,mathescape=true,fontsize=\normalsize]{python}

	
	
	
	
def |$f_t$|(match, |$arg_1, ..., arg_n$|):
	
	|$rm_i.id$| = match.getbinding(|$rm_i.s$|)
	
	sequence = Sequence()
	
		
		# snip
		
		# snip
		
		# snip
		
	return sequence
\end{minted}
\caption{Code generation for \TermSequenceNoArg.}
\label{codegen-term-sequence}
\end{figure}

Further code is generated based on the type of $t_i$. There are a few cases to consider.

\begin{itemize}
\item
The element is \TermRepeat. Retrieve the \texttt{ForEach} annotation and let $fe_1, ..., fe_n$ be the resulting set of pattern-variables. Note that these pattern-variables may be read from the match or simply passed as function parameters. First, ensure that the terms that are assigned to these pattern-variables are \texttt{Sequence} instances with equal lengths. An Exception is raised if the test fails. Otherwise, iterate over terms assigned to $fe_1, ..., fe_n$, retrieve the term at position $i$ and call $f_{t_i}$.

\begin{figure}
\begin{minted}[tabsize=2,obeytabs,escapeinside=||,mathescape=true,fontsize=\normalsize]{python}

lengths = set([])
	
	
assert isinstance(|$fe_i$|, Sequence)
lengths = lengths.union(|$fe_i$|.length())
assert len(lengths) == 1
	
for j in range(list(lengths)[0]): # convert set to list
	
	|$fe_i^{j}$| = |$fe_i$|.get(j)
	{% endfor %)
	sequence.append(|$f_{t_i}$|(match, |$fe_1^{j}, ..., fe_1^{j}$|))
\end{minted}
\caption{Code generation for term-template \RepeatNoArg \space in \TermSequenceNoArg.}
\label{codegen-term-sequence-repeat}
\end{figure}

\item
The element is \PythonCall. Call $f_{t_i}$ to obtain the term $t^{\prime}$. Notice that arguments for $f_{t_i}$ are generated using "InArg" annotations of $f_{t_i}$.
	\begin{enumerate}
	\item Insertion $mode$ is \texttt{Normal}. Append $t^{\prime}$ to the sequence.
	\item Insertion $mode$ is \texttt{Splice}. Ensure that $t^{\prime}$ is \texttt{Sequence} and then append each term in $t^{\prime}$ to the \texttt{term} \texttt{Sequence}.
	\end{enumerate}

\begin{figure}
\begin{minted}[tabsize=2,obeytabs,escapeinside=||,mathescape=true,fontsize=\normalsize]{python}

	|$t^{\prime}$| = |$f_{t_i}$|(match)
	
	sequence.append()
	
	assert isinstance(|$t^{\prime}$|, Sequence)
	for e in |$t^{\prime}$|:
		sequence.append(e)
	
#snip
\end{minted}
\caption{Code generation for term-template \PythonCallNoArg \space in \TermSequenceNoArg}
\label{codegen-term-sequence-pycall}
\end{figure}

\item Otherwise, simply call $f_{t_i}$ and append results to the sequence. Notice that the arguments for $f_{t_i}$ are generated using "InArg" annotations of $f_{t_i}$.

\begin{figure}
\begin{minted}[tabsize=2,obeytabs,escapeinside=||,mathescape=true,fontsize=\normalsize]{python}

	
t = |$f_{t_i}$|(match, |$a_1, ..., a_n$|)
sequence.append(t)

\end{minted}
\caption{Code generation for any other term-template in \TermSequenceNoArg}
\label{codegen-term-sequence-anyother}
\end{figure}
\end{itemize}

\subsection{Repeat}
Given \TermRepeat, generate the function $f_{t_r}$ for $t_r$. Since repetitions are handled in \TermSequenceNoArg, no special code generation is required here.

\subsection{InHole}
Given $t=$ \TermInHole, generate $f_{p_1}$ and $f_{p_2}$ for term-templates $p_1$ and $p_2$. Call $f_{p_1}$ and $f_{p_2}$ with appropriate arguments and then call \texttt{plughole} function as described in Section \ref{section:runtime-terms}.

\begin{figure}
\begin{minted}[tabsize=2,obeytabs,escapeinside=||,mathescape=true,fontsize=\normalsize]{python}

	
	
	
def |$f_t$|(match, |$arg_1, ..., arg_n$|):
	
	
	|$t_1^{\prime}$| = |$f_{t_1}$|(match, |$a_1, ..., a_n$|)
	|$t_2^{\prime}$| = |$f_{t_2}$|(match, |$b_1, ..., b_n$|)
	return plughole(|$t_1^{\prime}$|, |$t_2^{\prime}$|)
\end{minted}
\caption{Code generation for \PatternInHoleNoArg \space term-templates.}
\label{codegen-term-inhole}
\end{figure}


\subsection{PythonCall}
Given $t$ =\PythonCall \space generate functions $f_1$, ..., $f_n$ for $t_1$, ..., $t_n$. Call procedures $f_{t_1}$, ..., $f_{t_n}$ with appropriate arguments and let $t_1^{\prime}$, ..., $t_1^{\prime}$ be the resulting terms. Call $pyfunc$ with the arguments $t_i^{\prime}$. Note that $t$ has neither \texttt{InArg} nor \texttt{MatchRead} annotations, by construction.

\begin{figure}
\begin{minted}[tabsize=2,obeytabs,escapeinside=||,mathescape=true,fontsize=\normalsize]{python}

	
	
def |$f_t$|(match):
	
	|$t_i^{\prime}$| = |$f_{t_i}$|(match)
	
	return |$pyfunc$|(|$t_1^{\prime},...,t_n^{\prime}$|)

\end{minted}
\caption{Code generation for \PythonCallNoArg \space term-templates.}
\label{codegen-term-pycall}
\end{figure}


\subsection{MetafunctionApplication}
The metafunction application is very similar to Python calls, except that there's only one term-template to be passed as an argument to the metafunction instead of many.

Given the term-template $t=$ \ApplyMetafunction, generate the function $f_t$ for $t$. Then, generate the function $f_{t_m}$ for term-template $t_m$. Call $f_{t_m}$ with the appropriate arguments, and feed results of $f$ into the metafunction with name $mf$.

\begin{figure}
\begin{minted}[tabsize=2,obeytabs,escapeinside=||,mathescape=true,fontsize=\normalsize]{python}

	
	{% |$f_{t_r}$| = codegen(|$t_r$|) %]
	
def |$f_t$|(match):
	|$t_r^{\prime}$| = |$f_{t_i}$|(match)
	
	return |$mf$|(match, |$a_1, ..., a_n$|)

\end{minted}
\caption{Code generation for \MetafunctionApplicationNoArgs \space term-templates.}
\label{codegen-term-mfapply}
\end{figure}
