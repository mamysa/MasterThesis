\section{Term-Templates}

\subsection{Function Signature and Match Reads}

Recall that `Term\_EllipsisDepthChecker` annotates term template with three kinds of annotations:
\begin{itemize}
\item
* `InArg(paramname)` is used to generate a signature of term-building procedure; value of `paramname` is included as parameter to the procedure. All term-generating functions include `match` parameter at position one.

\item
* `ReadMatch(paramname, matchsym) indicates that there's a sequence of child term-building procedures that require a term assigned to `matchsym` in `Match` object. This means that that term-generating function must include assignment of \texttt{match.getbinding(matchsym)} to \texttt{paramname}.

\end{itemize}

\subsection{Literal Terms}

Given \TermLiteral, create new term based on $l$ and initialize it with $v$.


\subsection{Pattern Variables}

Given $t =$ \PatternVariable, one of two scenarios may happen. 
\begin{itemize}
\item $t$ has InArg(paramname) annotation, function simply returns paramname.
\item $t$ has ReadMatch annotation, meaning value of match.getbinding(sym) must be returned.
\end{itemize}

\subsection{Term Sequence}

* First, for each `ReadMatch` annotation, produce assignment as described above. Now, we need to call term-building procedures of each element in the term-sequence-template. There are a few cases to consider here. 

\begin{itemize}
	\item
	Element is \TermRepeat. Generate code for term-template $t$ and let $f$ be resulting function.  Retrieve ForEach(param) annotations. Ensure that terms bound to each param are term sequences that have the number of elements $n$. 
	For each term $param_1$, $param_2$, ..., $param_n$, retrieve term at position $i$ and call function $f$. Returned term is appended to the term sequence.
	\item
	Element is \PythonCall. Recursively generate code for it and let $f$ be resulting functrion. Then, there are two cases to consider. (TODO generating arguments?)
	\begin{enumerate}
	\item Insertion mode is normal. Simply call $f$ with required arguments and append results to the sequence.
	\item Insertion mode is splice. Call $f$ with required arguments and let $s$ be the term produced by $f$. Ensure that $s$ is a sequence by raising an Exception if it is not. Append each term $s_{child} \in s$ to the sequence.
	\end{enumerate}
	\item Otherwise, simply generate function $f$ for any other term and call $f$ with appropriate arguments. 
	\end{itemize}


\subsection{InHole}
Generate procedures for both term-templates. After calling them with appropriate arguments, `plughole` function is called. It locates `hole` in the first term and replaces it with the second term and returns resulting term.

\subsection{Python Function Calls}
Given \PythonCall generate functions $f_1$, ..., $f_n$ for $t_1$, ..., $t_n$. Call procedures $f_1$, ..., $f_n$ with appropriate arguments and let $o_1$, ..., $o_n$ be resulting terms. Call $function$ with $o_1$, ..., $o_n$.

\subsection{Metafunction Application}
Given \ApplyMetafunction, generate function $f$ for term-template $t$. Call $f$ with appropriate arguments, and feed results of $f$ into metafunction with name $n$.


