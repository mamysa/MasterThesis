\section{Notation}

Few words about notation. The original intent was to simply reuse actual Python code but it turned out to be not particularly space-friendly. Instead, to stay consistent with notation described in Chapter TODO, the following notation for code-generation is used, as seen in code snippet below.


\begin{minted}[tabsize=2,obeytabs,escapeinside=||,mathescape=true,fontsize=\small]{python}

	
	
def |$f_p$|(match, term, head, tail):
	tmp0 = |$isa_p$|(term)
	
	if tmp0 == True:
		tmp1 = match.addtobinding(|$s$|, term) 
		head = head + 1
		return [(match, head, tail)]
return []


\end{minted} 

\begin{itemize}
\item Statements between \texttt{bla} and \texttt{bla} are Python statements executed by the compiler. Statements that are not between is the generated Python code.
\item Variables such as $f_p$ are \textit{meta-variables}; when meta-variable is used in description of generated Python code, actual value assigned to it will be emitted instead. 
\item Special \texttt{endfor}, \texttt{enddef}, \texttt{endif} are used to indicate an end of Python block.
\item Occasionally, there's a need for iterative Python statement generation, based on contents of PAttern or TermTemplate. Array reference is used to retrieve fields of compile-time objects.
\end{itemize}
