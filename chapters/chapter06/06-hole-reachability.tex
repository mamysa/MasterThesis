\section{Hole Reachability}

Recall that pattern \InHolePattern requires $p_1$ to match exactly one \lstinline{hole} term. The main problem is counting a number of holes matched by non-terminal symbols of the language. Consider the language \lstinline{MatchesManyHoles} listed below.

\begin{lstlisting}
(define-language MatchesManyHoles
	(n ::= number)
	(P ::= E (n ... E ...))
	(E ::= (+ E P) (+ n E) hole))
\end{lstlisting}

To compute value for non-terminal \lstinline{E} analysis of pattern \lstinline{(+ E P)} is required but the vallue of \lstinline{E} is not yet known. Moreover, computing a single value that represents number of holes in insufficient; pattern lstinline{(E ...)} may match zero or many holes. Thus, a \textbf{minimum} and \textbf{maximum} number of holes matched by some pattern must be computed. 

\subsection{Graph Modelling}

Graph consists of two parts - \textbf{outer-nodes} and \textbf{inner-nodes}.

Outer-nodes are a stand-in for the pattern and come in four different varieties:

\begin{itemize}
\item \lstinline{Sequence} representing pattern-sequences.
\item \lstinline{Repeat} representing ellipses.
\item \lstinline{LeafHole} that represents pattern \lstinline{hole}.
\item \lstinline{LeafNotHole} that represents any other pattern except \InHolePattern
\end{itemize}

Inner-nodes are stored in \lstinline{Sequence} and \lstinline{Repeat} outer-nodes and act as non-terminal references. Each inner-node has a set of outer-node successors pointing to a set of expressions that the non-terminal matches. 

Graph for language defined above can be seen in the following figure: 
TODO

\subsection{Graph Construction}

Constructing the graph involves two stages: 
\begin{enumerate}
\item
Creation of outer nodes for each top-level pattern such as `(+ E P)` or `hole`.
\item
Creation of inner nodes and edges from inner to outer nodes. If outer node is a `Sequence` and it contains other sequences or ellipses, additional creation of outer-nodes will be required. 
\end{enumerate}


\subsection{Creation of Outer Nodes}

Need to maintain a mapping `M` between patterns in non-terminal definition. For each non-terminal definition `N` let `E\_N = \{\}` be a set storing top-level non-terminal references. For example, in the language described above for non-terminal definition `P` the only such reference is `E`.

Each non-terminal definition `N` is visited and we may come across the following patterns.
\begin{itemize}

\item
\BuiltInPattern. If kind is `Hole` then `n = LeafHole` node is created.  Otherwise, `n = LeafNotHole` node is created. Thus, `M = M union {(p, n)}`.
\item
\Nt. This means that `N` is also `nt` and thus all expressions in non-terminal definition of `nt` should also be reachable from any inner-node that reaches expressions in `N`. Thus, `E\_N = E\_N union {nt}`.
\item
\InHolePattern. Raise compilation error and quit. This, however, is not actual behaviour of the PltRedex. (TODO explain why we handle it differently?)
* `p = PatternSequence(pattern ...)`. If length of `p` is zero, then `n = LeafNotHole`, otherwise `n = Sequence`. Thus, `M = M union {(p, n)}`.
\end{itemize}

Note that creating multiple `LeafHole` or `LeafNotHole` nodes is not necessary and existing nodes of these kinds can be reused. This means that the graph may have at most two leaf nodes.


\subsection{Creation of Inner Nodes and Edges}

The only top-level pattern kind that may contain inner nodes at this point is `Sequence` and thus they need to be visited. One thing that needs to be maintained is the stack `S` of outer-nodes where inner-nodes are to be stored.

For each non-terminal definition `N` and each pattern $p_i$, if $p_i$ is `PatSequence`, retrieve `M[$p_i$]` from the mapping and place on top of the stack S. For each subpattern $q_i$ in $p_i$, visit $q_i$.

\begin{itemize}
\item
\PatternSequence. Create outer-node `Sequence` `s` for `p`; create inner node `i`; add edge `i->s`. Retrieve parent outer-node `O` and add node `i` to it. Append `s` to the stack `S`. For each child pattern $c_i$ of `p`, recursively visit each $c_i$. Pop topmost element off the stack `S`.

\item
* \Repeat. Create outer-node `Repeat` `r` for `p`, create inner node `i`, add edge `i->r`. Retrieve parent outer-node `O` and add node `i` to it. Append `r` to the stack `S`. Visit child pattern `c` of `p`. Pop topmost element off the stack `S`. 

\item
\InHolePattern Raise compilation error and quit.

\item
\BuiltInPattern. Create inner node `i`. If kind is Hole then let s = LeafHole, otherwise s = LeafNotHole. Create edge `i->s`.  Retreieve parent outer-node `O` from the top of the stack and add node `i` to it.

\item
\Nt Create inner node `i`. For each pattern in non-terminal definition of `nt`, retrieve outer-node `o` from the mapping `M` and create edge `i->o`. Do the same for each non-terminal in the set `E\_nt`. Retrieve outer-node `O` from the stack and append `i` to it.

\end{itemize}

\subsection{Minimal/Maximal Value Representation and Arithmetic}

Minimum / maximum number of holes can be one of the following: `Zero`, `One`, `Many` (i.e. >= 2) and `Uninitialized` indicating said value should not be used for computation just yet.

Furthermore, need to define three operators - `+`, `min`, `max`. Assume Zero = 0, One = 1, Many = 2 and Uninitialized = 3

Addition operation.

\begin{align}
	a + b = 
	\begin{cases}
		3 \text{ if } a = 3 \text{ and } b = 3 \\
		a \text{ if } b = 3 \\
		b \text{ if } a = 3 \\
		min(a + b, 2) 
	\end{cases}
\end{align}

Min operation:

\begin{align}
	min(a,b) = 
	\begin{cases}
		3 \text{ if } a = 3 \text{ and } b = 3 \\
		a \text{ if } b = 3 \\
		b \text{ if } a = 3 \\
		min(a, b)
	\end{cases}
\end{align}

Max operation:
\begin{align}
	max(a, b) = 
	\begin{cases}
		3 \text{ if } a = 3 \text{ and } b = 3 \\
		a \text{ if } b = 3 \\
		b \text{ if } a = 3 \\
		max(a, b)
	\end{cases}
\end{align}


\subsection{Computation of Values}
Min/max values are computed both for inner-nodes and outer-nodes. First, they have to be initialized.

\begin{itemize}
\item
Inner nodes are intialized with (Uninitialized, Uninitialized)
\item
`Sequence` and `Repeat` outer nodes are initialized with (Uninitialized, Uninitialized)
\item
`LeafHole` outer node is initialized with (One, One)
\item
`LeafNotHole` outer node is initialized with (One, One)
\end{itemize}


Given initial values $min_inner$, $max_inner$ of some inner-node  and new values $v_1$, $v_2$, $min_inner = min(min_inner, v1)$, $max_inner = max(max_inner, v2)$.


Given outer-node `o`, there are a few cases to consider.

\begin{itemize}
\item
`LeafHole` and `LeafNotHole` - its min/max values are never updated.
\item
* `Sequence` - $min_o = \sum_{i in inner(o)} min_i$ and $max_o = \sum_{i in inner(o)} max_i$. Summation is described above.
\item
* `Repeat` - $min_o = Zero$, $max_o = Many if max_inner = One or max_inner = Many else Zero$.
\end{itemize}

\subsection{Hole Reachability Algorithm}

Before initiating graph traversal, should first verify if `LeafHole` node is actually in `V`. If there's no such node, that means the source pattern does not match any `hole` and thus each non-terminal should be annotated with `Zero,Zero` number of holes.

Otherwise, the graph is traversed in reverse direction starting from nodes `LeafHole` and `LeafNotHole` using breath-first strategy. Maintain queue Q that initially stores all `LeafHole` and `LeafNotHole` nodes.

The algorithm is as follows:
\begin{enumerate}
\item
Remove outer node `o` from the queue `Q`. 
\item
* For each inner-node `i` such that `i->o`:
	\begin{enumerate}
	\item
	if `parent(i) = o` then $update(i, o_min, o_max)$ and `update(o)`. What this means that if inner node contains an edge to the outer-node `o` `i` is a child of, thus can be considered a "self-loop". The reason for doing this will be discussed later. (TODO is this actually needed aside from really degenerate patterns?)
	\end{enumerate}

\item
For each inner-node `i` such that `i->o`:
	\begin{enumerate}
	\item
	if `parent(i) != o`, $update(i, o_min, o_max)$ and update(o). If values of $o_min, o_max$ are different from ones before update operation, add `o` to the queue `Q`.
	\end{enumerate}

\item
* Repeat until queue `Q` is empty.

\end{enumerate}


\subsection{Computing Min/max values for non-terminal definition}

Given non-terminal definition `nt` and set of patterns `P` it matches, $min_nt = max_nt = Uninitialized$. Then, $min_nt = \min_{i in P} min_i$ and $max_nt = \max_{i in P} max_i$.Note that some parrern `i` in `P` may be Nt, in which case $min_i$ and $max_i$ for non-terminal definition `i` first. By ensuring that language contains no non-terminal cycles, such recursive procedure s guaranteed to terminate.

After completing all computations, each non-terminal definition `nt` is annotated with corresponding $min_nt, max_nt$.


\subsection{Edge cases}.

This algorithm doesn't one very specific thing - infinite terms and presense of `hole`. Consider the language below

\begin{lstlisting}
(define-lagnuge Infinite
     (P ::= (E))      
     (E ::= P hole))
\end{lstlisting}

\subsection{Example}
TODO
