\section{Pattern: Pattern Variable Extraction}

Due to how code generation works, certain pattern elements require information about pattern-variables in child patterns rooted at that pattern. The algorithm traverses the pattern, locates all pattern-variables and annotates required nodes accordingly.
\subsection{Annotation}

New annotation kind is introduced: PatternVariableSymbols.

\subsection{Algorithm}

The pattern is visited recursively. Every time a set $S$ containing pattern variables is returned.

\begin{itemize}
\item 
\Nt and \BuiltInPattern: Return $\{ s \}$.

\item
\LiteralPattern Return empty set.

\item $p_{in}$ = \PatternSequence  Let $pv_i^{\prime}$ be the set of pattern variables after recursively visiting $p_i$. The set of pattern variables for $p_{in}$, $pv_{in}$ is computed in the following way: $pv_{in} = \bigcup_{i=1}^{n}pv_i^{\prime}$

\item $p_{in}$ = \Repeat let $pv_i$ be the set of pattern variables after \Visit{$p$} . Annotate $p_{in}$ with $pv_i$.

\item $p_{in}$ = \InHolePattern. Let $pv_{in}=$ \Visit{$p_1$} $\cup$ \Visit{$p_2$}. Annotate $p_{in}$ with $pv_{in}$.
\end{itemize}

This completes pattern traversal. Let $p$ the topmost pattern element (i.e. it has no parent) and let $pv=$ \Visit{$p$}. Annotate $p$ with $pv$.

\section{Example}
TODO
