\section{Pattern: \lstinline{in-hole} Pattern Checker}

Having computed `min/max` numbers for each non-terminal in the language, now the question of whether `pattern1` in pattern \InHolePattern actually matches exactly one hole. Doing so involves two stages: 

\begin{enumerate}
\item
Locating \InHolePattern in the top-level pattern.
\item
Checking if $p_1$ matches exactly one hole. Since $pattern1$ doesn't necessarily have to be a non-terminal, extra work will have to be required.
\end{enumerate}

\subsection{Checking pattern for number of holes}

Assuming \InHolePattern pattern has already found, $p_1$ has to be visited recursively. One of the following nodes may be visited.

\begin{itemize}
\item
\BuiltInPattern - if $t=$hole then $min_p = One, max_p = One$, otherwise $min_p = Zero, max_p=Zero$.

\item
\Nt. Read appropriate annotation from non-terminal definition in `define-language` and return values.

\item

\LiteralPattern min\_p = Zero, max\_p = Zero.

\item
\InHolePattern. Return $min_p = Zero, max_p = Zero$, this seems to be default PltRedex behaviour.

\item
\Repeat. Recursively visit the $p$ obtaining $min_p2$, $max_p2$. The logic for Repeat is identical to the one used in **Hole Reachability**; that is:
	\begin{itemize}
	\item
	$min_n = Zero$
	\item
	$max_p = Many if max_p2 = One or max_p2 = Many$
	\item
	$max_p = Zero$ otherwise
	\end{itemize}

\item
* \PatternSequence. Initially, $min_p$, $max_p = Unitialized$. Then, $min_p = \sum_{p_i in p} minp_i$. To obtain $minp_i$, $p_i$ has to be recursively visited.
\end{itemize}

\subsection{Example}
TODO
